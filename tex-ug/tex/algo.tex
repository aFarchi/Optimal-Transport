
\section{Algorithms\label{ssec:algorithms}}
\noindent

    \subsection{Primal-Dual algorithm}
    \noindent

        The \pdAlgo{} algorithm allows one to minimize a functional $\costToMin=\costIToMin+\costIIToMin\circ\linearOperator$ where 
        \linearOperator{} is a linear operator and \costIToMin{} and \costIIToMin{}
        are simple functions. \\

        Here we apply this method by considering the minimization space to be the set of discretized
        couples \coupleFM{} defined on staggered grids. 
        \linearOperator{} is the linear operator that interpolates a couple 
        \coupleFM{} defined on staggered grids into a couple \coupleFM{} defined on centered grids. It is implemented by 
        the \pcode{interpolation(self)} method of class \pcode{StaggeredField}.

        \costIToMin{} is the cost function related to the continuity constraint \eqref{eq:continuity} and the boundary conditions
        \eqref{eq:boundaries} and \eqref{eq:null-flux} -- therefore when using this algorithm, one needs to select 
        the \pcode{ProxCdivb} version of the \pcode{proxCdiv} proximal operator. See section \ref{sssec:proximal-selection}
        for more details about the selection of the versions of the proximal operators.

        \costIIToMin{} is the cost function related to the cost of tranport \eqref{eq:cost}.\\

        One then defines three recurrent sequences \pdStateU{n}, \pdStateY{n} and \pdStateV{n} that are sequences of couples \coupleFM{}
        defined on staggered grids for \pdStateU{n} and \pdStateY{n} and on centered grids for \pdStateV{n} by:
        \begin{align}
            \pdStateV{n+1} &= \pdParSigma \cdot 
                                \proximal{\costIIToMin{}/\pdParSigma}{\frac{1}{\pdParSigma}\cdot\pdStateV{n}+\linearOperator\left(\pdStateY{n}\right)}, \\
            \pdStateU{n+1} &= \proximal{\pdParTau\costIToMin{}}{\pdStateU{n}-\pdParTau\cdot\linearOperator\left(\pdStateV{n+1}\right)}, \\
            \pdStateY{n+1} &= \left( 1 + \pdParTheta \right ) \cdot \pdStateU{n+1} - \pdParTheta \cdot \pdStateU{n}.
        \end{align}

        If $\pdParTheta\in[0,1]$ and $\pdParSigma\pdParTau<1$ then \pdStateU{n} converge towards the solution of the 
        discretized minimization problem.

    \subsection{Douglas-Rachford algorithm for three cost functions}
    \noindent

        The \drAlgo{} allows one to minimize a functional \costToMin that is the sum of a finite number of simple
        functionals. Here we use it to minimize the sum of $3$ functionals:
        $\costToMin=\costIToMin+\costIIToMin+\costIIIToMin$.

        We consider here the minimization space to be the space of the couples of discretized couples \coupleFM{},
        the first couple defined on the staggered gris and the second couple defined on the centered grids.

        \costIToMin{} is
        the cost of transport \eqref{eq:cost} and the continuity constraint \eqref{eq:continuity}. The continuity
        constraint only applies to the first couple \coupleFM{} (defined on the staggered grid) whereas the cost of transport
        only applies to the second couple \coupleFM{} (defined on the centered grid).

        \costIIToMin{} is the interpolation constraint, that mixes both couples \coupleFM{}.

        \costIIIToMin{} is the boundary conditions constraint \eqref{eq:boundaries} and \eqref{eq:null-flux}. It only
        applies to the first couple \coupleFM{} (defined on the staggered grid).\\

        Note that, as long as each boundary condition required for the optimal transport problem appear at least in one functional, 
        one can freely choose any variation of the proximal operators for the constraints. See section \ref{sssec:proximal-selection}
        for more details about the way proximal operators are selected.\\

        One then defines four recurrent sequences \drTStateUI{n}, \drTStateUII{n}, \drTStateUIII{n} and \drTStateX{n} that
        are in the minimization space (\ie{} that are couples of couples \coupleFM{}) by:
        \begin{align}
            \drTStatePI &= \proximal{\drTParGamma\costIToMin}{\drTStateUI{n}},\\
            \drTStatePII &= \proximal{\costIIToMin}{\drTStateUII{n}},\\
            \drTStatePIII &= \proximal{\costIIIToMin}{\drTStateUIII{n}},\\
            \drTStateP &= \drTParOmegaI \cdot \drTStatePI + \drTParOmegaII \cdot \drTStatePII + \drTParOmegaIII \cdot \drTStatePIII,\\
            \drTStateUI{n+1} &= \drTStateUI{n} + \drTParAlpha \cdot \left( 2 \cdot \drTStateP - \drTStateX{n} - \drTStatePI \right),\\
            \drTStateUII{n+1} &= \drTStateUII{n} + \drTParAlpha \cdot \left( 2 \cdot \drTStateP - \drTStateX{n} - \drTStatePII \right),\\
            \drTStateUIII{n+1} &= \drTStateUIII{n} + \drTParAlpha \cdot \left( 2 \cdot \drTStateP - \drTStateX{n} - \drTStatePIII \right),\\
            \drTStateX{n+1} &= \left( 1 - \drTParAlpha \right) \cdot \drTStateX{n} + \drTParAlpha \cdot \drTStateP,
        \end{align}
        where \drTStateP{}, \drTStatePI{}, \drTStatePII{} and \drTStatePIII{} are working variables in the minimization space.

        If $\drTParAlpha\in [0,1]$, $\drTParGamma>0$, $\drTParOmegaI>0$, $\drTParOmegaII>0$, $\drTParOmegaIII>0$ and 
        $\drTParOmegaI+\drTParOmegaII+\drTParOmegaIII=1$ then \drTStateX{n} converge towards the solution of the discretized
        minimization problem.

    \subsection{Douglas-Rachford algorithm for two cost functions}
    \noindent

        When using the standard boundary conditions (\ie{} no reservoirs), one can use the \pcode{ProxCdivb} version of the \pcode{proxCdiv}
        in the \drAlgo{} algorithm.
        The boundary constraint in functional \costIIIToMin{} are therefore redundant with those in functional
        \costIToMin{}. Hence, one can adapt the previous algorithm to use only the first two proximal operators. This is a clever
        way to reduce the computation complexity per iteration.

