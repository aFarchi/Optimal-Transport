
\section{Main objects defined}
\noindent

    \subsection{Discretization}
    \noindent

        In file \ppath{OT/OTObjects*D/grid/grid.py} are defined all the necessary classes to handle the discretized versions of
        the density and momentum fields on centered and staggered grids as defined in section 3.1 of \cite{farchi-2016} and in section 3 of
        \cite{papadakis-2014}.

    \subsection{Proximal operators}
    \noindent

        The proximal operators are defined in repertory \ppath{OT/OTObjects*D/proximals/}. Their computation mainly rely on the functions
        defined in the grid classes. For more details about the discretized versions of these operators, see section 4 of
        \cite{papadakis-2014}.\\

        Four different proximal operators are defined and
        that are referred in the code as \pcode{proxJ}, \pcode{proxCdiv}, \pcode{proxCsc} and \pcode{proxCb}.
        In table~\ref{tab:proximals-operators} we describe the contraint applied for each proximal operator and the state space
        of the input and output variable, \ie{} the grid on which the input and output variables must be defined to
        apply the proximal operator.

        
\begin{tablehtbp}
    
    \begin{tabular}{|l|l|l|l|}

        \hline
        Proximal operator & Sub-repertory  & Main constraint                                                    & State space of input and output      \\ \hline \hline
        \pcode{proxJ}     & \ppath{.}      & Cost function of transport \eqref{eq:cost}                         & Centered grid                        \\ \hline
        \pcode{proxCdiv}  & \ppath{div/}   & Continuity equation \eqref{eq:continuity}                          & Staggered grid                       \\ \hline
        \pcode{proxCsc}   & \ppath{sc/}    & Interpolation constraint                                           & Couple staggered grid, centered grid \\ \hline
        \pcode{proxCb}    & \ppath{bound/} & Boundary conditions \eqref{eq:boundaries} and \eqref{eq:null-flux} & Staggered grid                       \\ \hline

    \end{tabular}

    \caption{Details about the proximal operators.}
    \label{tab:proximals-operators}

\end{tablehtbp}
 

        For each constraint, there are variations of
        the projection operators that add (partial) boundary conditions to the main constraint.
        For example, in \ppath{div/}, \ppath{proxCdiv.py} defines
        the projection on the continuity constraint \eqref{eq:continuity}, \ppath{proxCdivb.py} defines the projection on the continuity constraint 
        \eqref{eq:continuity} with the boundary conditions \eqref{eq:boundaries} and \eqref{eq:null-flux} and
        \ppath{proxCdivtb.py} defines the projection on the continuity constraint \eqref{eq:continuity} with temporal boundary conditions
        \eqref{eq:boundaries}.
        See table~\ref{tab:proximals-variations} for more details.

        
\begin{tablehtbp}
    
    \begin{tabular}{|l|c|c|}

        \hline
        Class              & Boundary conditions                                                   & State space of input and output      \\ \hline \hline
        \multicolumn{3}{|c|}{Variations of the divergence constraint}                                                                     \\ \hline \hline
        \pcode{ProxCdiv}   & None                                                                  & Staggered grid                       \\ \hline
        \pcode{ProxCdivb}  & \eqref{eq:boundaries} and \eqref{eq:null-flux}                        & Staggered grid                       \\ \hline
        \pcode{ProxCdivtb} & \eqref{eq:boundaries}                                                 & Staggered grid                       \\ \hline \hline
        \multicolumn{3}{|c|}{Variations of the interpolation constraint}                                                                  \\ \hline \hline
        \pcode{ProxCsc}    & None                                                                  & Couple staggered grid, centered grid \\ \hline
        \pcode{ProxCscb}   & \eqref{eq:boundaries} and \eqref{eq:null-flux}                        & Couple staggered grid, centered grid \\ \hline
        \pcode{ProxCscrb}  & \eqref{eq:boundaries}, only at $\ficttime=0$ and \eqref{eq:null-flux} & Couple staggered grid, centered grid \\ \hline
        \pcode{ProxCsctb}  & \eqref{eq:boundaries}                                                 & Couple staggered grid, centered grid \\ \hline \hline
        \multicolumn{3}{|c|}{Variations of the boundary conditions}                                                                       \\ \hline \hline
        \pcode{ProxCb}     & \eqref{eq:boundaries} and \eqref{eq:null-flux}                        & Staggered grid                       \\ \hline
        \pcode{ProxCrb}    & \eqref{eq:boundaries}, only at $\ficttime=0$ and \eqref{eq:null-flux} & Staggered grid                       \\ \hline
        \pcode{ProxCtb}    & \eqref{eq:boundaries}                                                 & Staggered grid                       \\ \hline

    \end{tabular}

    \caption{Details about the variations of the proximal operators.}
    \label{tab:proximals-variations}

\end{tablehtbp}


        Choosing the correct version for the proximal operators is a critical step. See section \ref{sssec:proximal-selection}
        for more details about the way proximal operators are chosen.

    \subsection{Algorithms}
    \noindent

        The algorithms are implemented in repertory \ppath{OT/OTObjects*D/algorithms/}. When using this module to solve an 
        optimal transport problem, there are two steps: one has to construct
        an algorithm object, with the correct set of parameters, and then one just need to call the \pcode{run(self)} member of the object.
        The construction of the algorithm object requires a configuration object, which is in fact just a container object
        that contains the set of parameters.

        Three algorithms are implemented: the \pdAlgo{} algorithm in \ppath{pd/}, the \drAlgo{} algorithm for the minimization of two
        functionals in \ppath{adr/} and the \drAlgo{} algorithm for the minimization of three functionals in \ppath{adr3/}.
        More details about the algorithms are available in section \ref{ssec:algorithms}.

    \subsection{Configuration}
    \noindent

        The class for the configuration object is defined in \ppath{OT/OTObjects*D/configuration.py}. It inherits from a default
        configuration class that handles the interface with a configuration file.
        It also defines a member \pcode{algorithm(self)} that constructs the algorithm for a given configuration object.

