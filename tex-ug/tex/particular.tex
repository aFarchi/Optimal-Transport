
\section{Specific aspects}
\noindent
    
    \subsection{Relaunch simulations\label{sssec:relaunch-sim}}
    \noindent

        The \python file \ppath{reLaunchSimulation*D.py} can be used to relaunch a simulation. This script will construct
        the configuration from the previous configuration file if provided by the keyword argument \bcode{CONFIG\us{}FILE}
        or directly from the configuration object saved in the output directory. In the latter case, one has to specify the
        previous output directory with the keyword argument \bcode{OUTPUT\us{}DIR}. Optional arguments include
        \bcode{NEW\us{}ITER\us{}TARGET}, which specify a new value for the number of algorithm iterations to perform, and 
        \bcode{PRINT\us{}IO}.

        Use one of the following commands.
        \lstset{style=codebash}
        \begin{lstlisting}
$ ./reLaunchSimulation*D.py CONFIG_FILE=configFile.cgf NEW_ITER_TARGET=1000 PRINT_IO=True\end{lstlisting}
        \begin{lstlisting}
$ ./reLaunchSimulation*D.py OUTPUT_DIR=/previous/output/dir/ NEW_ITER_TARGET=1000 PRINT_IO=True\end{lstlisting}

        Also note that using this script, one necessary relaunch the simulation with the same algorithm. There are also ways
        the relaunch a simulation with a different algorithm, for example see section \ref{sssec:initialization}.

    \subsection{Anamorphosis\label{sssec:anamorphosis}}
    \noindent

        For $\dimension=1$, the solution of the optimal transport problem can be obtained with an analytical formula, and this
        is known as the anamorphosis transform. For more details about this transformation, see section 2.4 of \precite{farchi-2016}.
        An implementation of these formula is available with this solver. It has been written in the exact same way as the
        \drAlgo{} and the \pdAlgo{} algorithms in the file \ppath{OTObjects1D/algorithms/anamorph/anamorphAlgorithm.py}.

        To use this \inG{algorithm}, one must set \pcode{algoName=ana} in the configuration file and define the parameter
        \pcode{PDFError} that will be used as a relative tolerance value to compute the inverse of cumulative density functions.
        Unlike the other algorithms, it will ignore the parameter \pcode{iterTarget} since it is not an iterative
        algorithm.

    \subsection{Analyse\label{sssec:analyse}}
    \noindent

        After performing a simulation, one may want to analyse the convergence rate of the algorithm. Some tools are implemented
        to this end in the repertory \ppath{OTObjects*D/analyse/}. First, the output files will be opened. As described in section
        \ref{ssec:launch-algo}, it contains the value of the converging variable every \pcode{nModeWrite} algorithm iteration. It
        also contains the execution time at the current iteration -- in order to perform time analysis. Then some operators
        will be applied to the variable.\\

        In file \ppath{operators1.py} are defined a first set of operators: the $L^\infty$ norm of the $\left(\dimension+1\right)$-divergence of the field
        (\ie{} the error in the continuity constraint), the minimum value of the density field and some numerical variations of the cost
        function \eqref{eq:cost}. In file \ppath{operators2.py} the operators defined compare the current value of the field to the last
        value of the field (which is therefore assumed to be the solution of the minimization problem): the $L^\infty$ norm of the difference.

        The results are stored in the file \ppath{analyse.bin} in \pcode{outputDir} in binary format using the \cpck{} module as following:
        \begin{enumerate}
            \item the \numpy{} array of iteration numbers;
            \item the \numpy{} array of iteration times;
            \item a list of string representing the operator names;
            \item the \numpy{} two-dimensional array of the operation values, whose shape is:\\number~of~iterations~$\times$~number~of~operators.
        \end{enumerate}

        With these informations, everything is available to plot analyses of the run as a function of time or as a function of iteration number.
        See section \ref{sssec:plotting-animating} for more information about plots. Also note that there is a \python launcher dedicated
        to this analyse process: \ppath{analyseSimulation*D.py}, to use with one of the following commands.
        \lstset{style=codebash}
        \begin{lstlisting}
$ ./analyseSimulation*D.py CONFIG_FILE=configFile.cgf\end{lstlisting}
        \begin{lstlisting}
$ ./analyseSimulation*D.py OUTPUT_DIR=/output/dir/\end{lstlisting}

    \subsection{Plotting and animating\label{sssec:plotting-animating}}
    \noindent

        With this module, two submodules are provided for plotting and animating purpose in the \ppath{OTObjects*D/plotting/} and
        \ppath{OTObjects*D/animating} repertories. The operation of these submodules is approximately the same as the main module.
        Two specific configuration classes are implemented. As usual, these submodules come with the dedicated \python launchers
        \ppath{plotSimulation*D.py} and \ppath{animateSimulation*D.py} and their usage require a configuration file. The configuration
        files are specific to each submodule, examples are available: \ppath{OTObjects*D/plotting/plotting.cfg.example} and 
        \ppath{OTObjects*D/animating/animating.cfg.example}. Use the launchers with the following commands.
        \lstset{style=codebash}
        \begin{lstlisting}
$ ./plotSimulation*D.py CONFIG_FILE=plotting.cgf PRINT_IO=True\end{lstlisting}
        \begin{lstlisting}
$ ./animateSimulation*D.py CONFIG_FILE=animating.cgf PRINT_IO=True\end{lstlisting}

        Note that these submodules rely on \mpl{} to draw plot and on movie encoders (\eg{} \ffmpeg{} or \mencoder{}) to  
        save the animations in files.

    \subsection{Initialization\label{sssec:initialization}}
    \noindent

        The algorithm presented in section \ref{ssec:algorithms} converge, no matter which initial condition is chosen for
        the algorithm state. However, choosing a clever initial condition is a good strategy to achieve a better convergence
        without increasing the number of algorithm iterations. Yet, there has been no study about the influence of the
        initial condition in this particular optimal transport problem.\\

        In our solver, all algorithm use the same initialization function \pcode{initialStaggeredField(config)}
        defined in the file 
        \ppath{OTObjects*D/init/initialFields.py}. Basically, it just constructs the linear interpolation between the temporal
        boundary conditions \fI{} and \fII{}.\\

        Alternatively the initial condition can be obtained from the output of a previous simulation. This is driven by
        parameter \pcode{initial} of the configuration object:
        \begin{itemize}
            \item if \pcode{initial} is $1$, then the algorithm will look for the result of a simulation whose result is 
                located in \pcode{outputDir} to initialize its state;
            \item if \pcode{initial} is $2$, then it will look in the repertory \pcode{initialInputDir};
            \item if \pcode{initial} is $3$, it will first look in \pcode{outputDir} then in \pcode{initialInputDir}.
        \end{itemize}

        If no previous run is found, the initialization method will fall back to the \pcode{initialStaggeredField(config)} function.
        With this method, an algorithm can catch the state of an other algorithm. In that case, it uses the converging
        variable to define its initial state.

    \subsection{Proximal variations selection\label{sssec:proximal-selection}}
    \noindent

        Choosing the correct versions for the proximal operators is a critical step while constructing the algorithms. 
        The proximal used by each algorithm is detailed in section \ref{ssec:algorithms}. When choosing the proximal operators,
        one must insure that each constraint appear at least in one proximal operator that will be used by the algorithm.\\

        In file
        \ppath{OTObjects*D/proximals/defineProximals.py}, the function \pcode{proximalForConfig(config)} defines the correct proximal
        operators according to the parameter \pcode{dynamics} of the
        configuration.
        \begin{itemize}
            \item If \pcode{dynamics} is $0$, then boundary conditions in space and time are added to all constraint.
            \item It is also the case if \pcode{dynamics} is $1$, but this time the spatial boundary conditions are supposed to be zero
                -- while they can be non-zero with \pcode{dynamics} equal to $0$, as presented in section \ref{sssec:bc-reservoir}.
            \item If \pcode{dynamics} is $2$, then the temporal boundary conditions are added to all constraints but no spatial boundary condition
                is applied. For this reason, the convergence of the algorithm with this method is questionable, as the 
                mathematical problem is not well defined.
        \end{itemize}

        The $3$ and $4$ values for \pcode{dynamics} are dedicated to the version of the problem with reservoirs (see section
        \ref{sssec:bc-reservoir}). The divergence constraint never include the boundary conditions (because it would break
        the symmetry of the problem and prevent one to use the Fourier transformation methods). They are included in the 
        interpolation constraint if \pcode{dynamics} is equal to $3$ but not if it is equal to $4$.\\

        As a consequence, the \pdAlgo{} algorithm must be used with \pcode{dynamics} equals to $1$ or $2$. The
        \drAlgo{} algorithm with two proximal operators 
        can be used with \pcode{dynamics} equals to $1$ or $2$ for the classical optimal transport problem
        and \pcode{dynamics} equals to $3$ for the optimal transport problem with reservoirs.
        The \drAlgo{} algorithm with three proximal operators can be used with \pcode{dynamics} equals 
        to $1$ or $2$ for the classical optimal transport problem and \pcode{dynamics} equals to $3$ or $4$ for 
        the optimal transport problem with reservoirs. Using  \pcode{dynamics} equals to $2$ is not recommanded.

    \subsection{Boundary conditions and reservoir approach\label{sssec:bc-reservoir}}
    \noindent
        
        \subsubsection{Boundary conditions}
        \noindent

            In the very definition of the optimal transport problem, we defined \fI{} and \fII{} to be probability
            density functions over \spaceE{}. In particular, this means that:
            \begin{equation}
                \int_{\spaceE} \fI\left(\x\right)\dx = \int_{\spaceE}\fII\left(\x\right)\dx = 1,
            \end{equation}
            which is a necessary condition to have a couple \coupleFM{} satisfying \eqref{eq:continuity}, \eqref{eq:boundaries}
            and \eqref{eq:null-flux}.
            In fact, this condition can be relaxed if one changes the null-flux condition \eqref{eq:null-flux} into:
            \begin{equation}
                \forall \coupleXT \in \partial\EtimesT, \quad \mom\coupleXT = \mom_0\coupleXT,
            \end{equation}
            with the new mass conservation condition:
            \begin{equation}
                \label{eq:mass-conservation}
                \int_{\partial\EtimesT} \left(\mom_0\coupleXT \cdot \dx \right) \dt = \int_{\spaceE}\left( \fII \left(\x\right) - \fI \left(\x\right)\right)\dx.
            \end{equation}

            Numerically, it doesn't change anything in the computation of the proximal operators, it just changes
            the construction of the numerical boundary conditions. These conditions are constructed during the
            construction of the configuration object by the function \pcode{boundariesForConfig(config)} of file
            \ppath{OTObjects*D/boundaries/defineBoundaries.py}. This function relies on the parameter 
            \pcode{boundaryType} of the configuration object. If \pcode{boundaryType} takes value 1 to 8, then
            default boundary conditions will be applied (see section \ref{sssec:examples}). 
            If \pcode{boundaryType} is 0, then the boundary conditions
            will be constructed with data from files. The file names for the \fI{} and \fII{} conditions are determined
            by the parameters \pcode{filef0} and \pcode{filef1} of the configuration object. If necessary -- \ie{} if
            the parameter \pcode{dynamics} is 0 as said in section \ref{sssec:proximal-selection} -- spatial boundary conditions 
            are also determined from files \pcode{filem0} and \pcode{filem1} for the $\dimension=1$ case and from files
            \pcode{filemx0}, \pcode{filemx1}, \pcode{filemy0} and \pcode{filemy1} for the $\dimension=2$ case.
            The corresponding boundary conditions are presented in table \ref{tab:boundary-files}.

            
\begin{tablehtbp}
    
    \begin{tabular}{|c|c|}

        \hline
        Parameter to define & Boundary condition driven \\ \hline \hline
        \pcode{filef0}      & $\ficttime=0$             \\ \hline
        \pcode{filef1}      & $\ficttime=1$             \\ \hline \hline
        \multicolumn{2}{|c|}{$\dimension=1$}            \\ \hline \hline
        \pcode{filem0}      & $\x=0$                    \\ \hline
        \pcode{filem1}      & $\x=1$                    \\ \hline \hline
        \multicolumn{2}{|c|}{$\dimension=2$}            \\ \hline \hline
        \pcode{filemx0}     & $\x_1=0$                  \\ \hline
        \pcode{filemx1}     & $\x_1=1$                  \\ \hline
        \pcode{filemy0}     & $\x_2=0$                  \\ \hline
        \pcode{filemy1}     & $\x_2=1$                  \\ \hline

    \end{tabular}

    \caption{Boundary conditions driven by the files.}
    \label{tab:boundary-files}

\end{tablehtbp}


            The format of the file accepted by the algorithm is detailed in section \ref{sssec:input}.
            After loading the fields from the files, one has to check that the mass is conserved according
            to \eqref{eq:mass-conservation}. See section \ref{sssec:normalization} for more information about normalization.
            If one is interested in solving the optimal transport problem without the mass conservation constraint,
            then one should consider using the reservoir method we developed and presented in \cite{farchi-2016}.

        \subsubsection{Reservoir approach}
        \noindent

            For this method, \spaceE{} is expanded by adding a reservoir variable at each boundary location of \spaceE{}.
            To keep control of the total mass inside the domain, we impose the null-flux condition \eqref{eq:null-flux}
            to the extended domain. Potentially inverting the role of \fI{} and \fII{} -- this operation is
            tracked by the variable \pcode{swappedInitFinal} of the configuration object -- one can assume that 
            \fI{} has more mass than \fII{}. Then we can impose the following boundary conditions:
            \begin{itemize}
                \item at $\ficttime=0$, all reservoirs must be empty, \ie{} all reservoir variables must be equal to zero;
                \item at $\ficttime=1$, the total mass inside the reservoir must be equal to the difference of mass between
                    \fI{} and \fII{};
                \item in the interior of \spaceE{} we impose the classical temporal boundary conditions \eqref{eq:boundaries}.
            \end{itemize}

            To solve the optimal transport problem with reservoir, one must use the \drAlgo{} algorithm with
            the parameter \pcode{dynamics} equals to $3$ or $4$ as presented in section \ref{sssec:proximal-selection}.
            Note that in this case, the boundary values of the arrays specified for \fI{} and \fII{} will
            be ignored and replaced by zero.

    \subsection{Input\label{sssec:input}}
    \noindent

        \subsubsection{Boundary conditions}
        \noindent

            The boundary conditions can be defined from files as presented in section \ref{sssec:bc-reservoir}. The files
            are read with the \pcode{arrayFromFile(fileName)} function, implemented in the file \ppath{utils/io/io.py}.
            It accepts either \ppath{.npy} files, loaded with with the \pcode{numpy.load(fileName)} function, or
            binary files, loaded with the \pcode{numpy.fromfile(fileName)} function and then reshaped with the
            \pcode{numpy.ndarray.reshape(newShape)} function to match the dimensions imposed by the parameters
            \pcode{M}, \pcode{N} and \pcode{P}.\\

        \subsubsection{Initial condition}
        \noindent

            As seen in section \ref{sssec:initialization}, the initial condition can be retrieved from a previous run.
            In that case, the algorithm will read the files \ppath{runCount.bin} -- to check that there has been indeed
            a run -- and \ppath{finalState.bin} that have both been written by an algorithm, according to the method
            presented in section \ref{sssec:output}.

    \subsection{Output\label{sssec:output}}
    \noindent

        \subsubsection{During the algorithm run}
        \noindent

            As mentioned in section \ref{ssec:launch-algo}, every \pcode{nModWrite} iterations, the algorithm writes
            the converging variable and the current time elapsed to a file.
            The file is named \ppath{states.bin} and is located in the \pcode{outputDir} repertory.
            The variable is written as a \pcode{StaggeredField} object, in binary format using the \cpck{} module.
            Note that the results are appended to the file \ppath{states.bin} so that one can keep the
            states of the previous runs.

        \subsubsection{At the end of the run}
        \noindent

            After the run, a full copy of the algorithm state is written in \pcode{outputDir}. More precisely,
            the configuration object is written in the file \ppath{config.bin} and the algorithm state in
            the file \ppath{finalState.bin}. Once again, the objects are written in binary format with the
            \cpck{} module. Note that the configuration is appended to the file in order to keep track
            of the configurations for the previous runs.

            Finally, the number in the file \ppath{runCount.bin} is incremented if it already exists. Else
            the number $1$ is written in file \ppath{runCount.bin}. This way, one is able to keep track of
            the number of runs performed.

    \subsection{Default examples\label{sssec:examples}}
    \noindent

        As seen in section \ref{sssec:bc-reservoir}, there are a few default configurations for the boundary conditions
        provided for testing purpose. They are all defined in the repertory \ppath{OTObjects*D/boundaries/} and the
        selection of the correct boundary condition happens in function \pcode{boundariesForConfig(config)} from file
        \ppath{defineBoundaries.py} according to the parameter \pcode{boundaryType}. The possibilities are described in
        table \ref{tab:boundary-examples}.

        
\begin{tablehtbp}
    
    \begin{tabular}{|c|l|l|}

        \hline
        \pcode{boundaryType} & File                     & Descrption                                       \\ \hline \hline
        $1$                  & \ppath{gaussian.py}      & One Gaussian                                     \\ \hline
        $2$                  & \ppath{gaussian.py}      & Sum of two Gaussians                             \\ \hline
        $3$                  & \ppath{gaussianSplit.py} & Sum of two Gaussians rejoining into one Gaussian \\ \hline 
        $4$                  & \ppath{gaussianSplit.py} & One Gaussian splitting into two Gaussians        \\ \hline
        $5$                  & \ppath{gaussianSine.py}  & One Gaussian with sine oscillations              \\ \hline
        $6$                  & \ppath{gaussianSine.py}  & One Gaussian with cosine oscillations            \\ \hline
        $7$                  & \ppath{custom.py}        & Guess what...                                    \\ \hline
        $8$                  & \ppath{custom.py}        & Guess what...                                    \\ \hline
        

    \end{tabular}

    \caption{Default boundary conditions depending on the parameter \pcode{boundaryType}.}
    \label{tab:boundary-examples}

\end{tablehtbp}


    \subsection{Normalization\label{sssec:normalization}}
    \noindent

        After defining the correct boundary conditions, one has to check the mass conservation (if not using the reservoir method).
        This is performed by the \pcode{normalize(normType)} method of class \pcode{Boundaries} whose behavior only depends on
        the \pcode{normType} parameter of the configuration object.

        \begin{itemize}
            \item If \pcode{normType} is $0$, then \fII{} is rescaled using the mass of \fI{}. 
            \item If \pcode{normType} is $1$, then \fI{} is rescaled using the mass of \fII{}.
            \item If \pcode{normType} is $2$, then \fI{} and \fII{} are not changed but the
                spatial boundary conditions are adapted to compensate for the mass losses. 
            \item If \pcode{normType} is $3$, then the mass of \fI{} and
                the mass of \fII{} are set to unity.
        \end{itemize}

