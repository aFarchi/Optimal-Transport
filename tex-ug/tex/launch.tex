
\section{Launching an algorithm\label{ssec:launch-algo}}
\noindent

    \subsection{With the default launchers}
    \noindent

        The easiest way to launch an algorithm is to use the default launchers provided in the top repertory.
        More precisely, the launcher to use is the \python file \ppath{launchSimulation*D.py} with the command:
        \lstset{style=codebash}
        \begin{lstlisting}
$ ./launchSimulation*D.py CONFIG_FILE=file.cgf PRINT_IO=True\end{lstlisting}
        where \ppath{file.cfg} is
        your configuration file and \bcode{PRINT\us{}IO} is an optionnal parameter that enables output on the screen.
        The launcher then reads your configuration file, construct the adequate configuration object, then constructs
        the algorithm object with the \pcode{algorithm(self)} method and make it run with the \pcode{run(self)} method.
        At the end, it save the results in the specified output repertory (see section \ref{sssec:output}) and make an analyse of the
        run (see section \ref{sssec:analyse}).\\

        An example of configuration file is given by \ppath{OT/OTObjects*D/OT*D.cfg.example}. It must define the following
        parameters:
        \begin{enumerate}
            \item \pcode{EPSILON}: variable used as tolerance value for some testings.
            \item \pcode{outputDir}: path to the repertory for 
                output\footnote{\NB{}: although \python{} supports relative paths, it is strongly recommanded to use absolte paths.}. See 
                section \ref{sssec:output} for more details about the output.
            \item \pcode{M}, \pcode{N} and \pcode{P}: discretization resolution along the dimensions in space for 
                \pcode{M} (only for the $\dimension{}=2$ case) and \pcode{N} and in time for \pcode{P}.
            \item \pcode{dynamics}: parameter used to select the correct variations of the proximal operators. 
                See section \ref{sssec:proximal-selection} for more details about the way proximal operators are selected.
            \item \pcode{boundaryType}: parameter used to select the correct boundary conditions. See section \ref{sssec:bc-reservoir} for 
                more information about the boundary conditions.
            \item \pcode{normType}: parameter used to select the correct normalization of the fields. See section \ref{sssec:normalization}
                for more details about normalization.
            \item \pcode{file**}: names of the files used to define the boundary conditions when needed. See section \ref{sssec:bc-reservoir} and
                section \ref{sssec:input} for more information about the boundary conditions and input.
            \item \pcode{algoName}: name of the algorithm to use. Use:
                \begin{itemize}
                    \item \pcode{pd} for the \pdAlgo{} algorithm;
                    \item \pcode{adr3} for the \drAlgo{} algorithm with three proximal operators;
                    \item \pcode{adr} for the \drAlgo{} with two proximal operators. 
                \end{itemize}
            \item \pcode{iterTarget}: number of algorithm iterations to run.
            \item \pcode{nModPrint}: while running the algorithm, display information every \pcode{nModPrint} algorithm iteration. Information include
                the number of iterations run, the time elapsed and the current value of the cost of transport \eqref{eq:cost}.
            \item \pcode{nModWrite}: while running the algorithm, every \pcode{nModWrite} algorithm iteration, the current value of the 
                converging variable will be written
                in the output directory. See section \ref{sssec:output} for more details about output.
            \item \pcode{initial}: parameter used to select the correct method to compute the initial condition. See section \ref{sssec:initialization}
                for more details about initialization.
            \item \pcode{initialInputDir}: when the initial condition is obtained from a previous run, this variable must contain the repertory 
                in which the results have been saved. See section \ref{sssec:initialization} and section \ref{sssec:input} for more
                informations about initialization and input.
        \end{enumerate}

        Also the algorithm parameters must be defined. See table \ref{tab:algo-parameters} for more details about the parameters
        specific to each algorithm.

        
\begin{tablehtbp}
    
    \begin{tabular}{|c|c|c|}

        \hline
        Parameter to define & Variable name  & Default value                  \\ \hline \hline
        \multicolumn{3}{|c|}{For the \pdAlgo{} algorithm}                     \\ \hline \hline
        \pdParSigma         & \pcode{theta}  & $85$                           \\ \hline
        \pdParTau           & \pcode{sigma}  & $1/85$                         \\ \hline
        \pdParTheta         & \pcode{tau}    & $1$                            \\ \hline \hline
        \multicolumn{3}{|c|}{For the \drAlgo{} with two proximal operators}   \\ \hline \hline
        \drTParGamma        & \pcode{gamma}  & $1/75$                         \\ \hline
        \drTParAlpha        & \pcode{alpha}  & $1.998$                        \\ \hline \hline
        \multicolumn{3}{|c|}{For the \drAlgo{} with three proximal operators} \\ \hline \hline
        \drTParGamma        & \pcode{gamma3} & $1/75$                         \\ \hline
        \drTParAlpha        & \pcode{alpha3} & $1.998$                        \\ \hline 
        \drTParOmegaI       & \pcode{omega1} & $0.33$                         \\ \hline
        \drTParOmegaII      & \pcode{omega2} & $0.33$                         \\ \hline
        \drTParOmegaIII     & \pcode{omega3} & $0.34$                         \\ \hline

    \end{tabular}

    \caption{Parameters to define in the config file.}
    \label{tab:algo-parameters}

\end{tablehtbp}


    \subsection{Alternatives}
    \noindent

        If one doesn't want to use the provided \python{} launchers, the best way is probably to define an (almost) empty configuration class
        and to make it contain every variable needed for the construction of the algorithm. We provide here a minimal example to solve the
        optimal transport between two one-dimensional Gaussians using
        the \pdAlgo{} algorithm.

        \lstset{style=codepython}
        \lstinputlisting{py/example_without_launcher.py}

